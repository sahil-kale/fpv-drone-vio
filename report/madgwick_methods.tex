\documentclass[bare_jrnl_transmag]{subfiles}
\begin{document}
\subsection{Madgwick Filter Implementation}
The Madgwick filter was implemented in Python to provide an orientation pose estimate. However, due to the nature of the drone's flight and real-world conditions, modifications were required to the original algorithm to ensure the filter was able to provide a stable orientation estimate. The modifications made to the Madgwick filter are described below. \newline

\subsubsection{Gyroscope Bias Compensation}
Unlike other inertial orientation filters that are capable of measuring bias in the gyroscopic measurements (ex: Multiplicative EKFs), a Madgwick filter is not capable of doing so. A Madgwick filter can implicitly correct for gyroscopic bias with the correction step, but it does so relatively poorly as the bias is not explicitly modelled. As a result, the VIO algorithm calculated the average gyroscopic bias on startup and is subtracted from the gyroscopic measurements within the Madgwick filter implementation before integrating them per Equation \ref{eq:madgwick_gyro_integration}. \newline

\subsubsection{High Acceleration Compensation}
A key limitation of a Madgwick filter is that it assumes the accelerometer measurements are only affected by gravity, and not by any other forces acting on the drone. This assumption is not valid in all cases, especially when the drone undergoing high acceleration maneuvers. When evaluating the implemented filter's performance in high acceleration scenarios on the tuning dataset, the orientation estimation was found to diverge significantly and increase the RMSE. Given the context of the dataset is an FPV drone where high acceleration maneuvers are common, the correction step size $\mu$ was made to be adaptive to the accelerometer measurements. The algorithm for the adaptive step size is shown below, with the parameters $x_1$, $x_2$, and $x_3$ being tuned heuristically. This change helped improve the orientation estimation in high acceleration scenarios. The algorithm is as follows:

\[
\mu_{\text{out}} =
\begin{cases}
0 & \text{if } \|\mathbf{a}\| \leq x_1 \\
\mu \cdot \dfrac{\|\mathbf{a}\| - x_1}{x_2 - x_1} & \text{if } x_1 < \|\mathbf{a}\| \leq x_2 \\
\mu \cdot \dfrac{x_3 - \|\mathbf{a}\|}{x_3 - x_2} & \text{if } x_2 < \|\mathbf{a}\| \leq x_3 \\
0 & \text{if } \|\mathbf{a}\| > x_3
\end{cases}
\]

\[
\text{where } x_1 = 4.905, \quad x_2 = 9.81, \quad x_3 = 14.715
\]

\end{document}