\documentclass[bare_jrnl_transmag]{subfiles}

\begin{document}
\subsection{Kalman Filter Theory}
A Kalman Filter is a type of algorithm which predicts the states of a system using their dynamics, and correcting the prediction with feedback from an external sensor. The filter is used to fuse the data coming from the IMU and the data from the camera. 
In the case of the project, the Kalman filter was used to predict the position of the drone relative to the world frame. The states of the Kalman Filter were thus 

\begin{equation} 
    [ t_x, t_y, t_z, v_x, v_y, v_z ] 
\end{equation}  which represent the translation and velocity of the drone in world frame. 
The inputs of the system were the IMU acceleration data and the camera distance output, while the output of the system was the updated position of the drone.\newline

The filter uses various noise and covariance matrices to add uncertainty to the model. P, Q, and R are the main matrices. The state covariance matrix (P), is responsible for representing the uncertainty in the system states. It essentially represents the trust in the state variables versus the measured values.
The Q and R covariance matrices represent two different noises. Q represents the process noise matrix, which models the uncertainty in the system dynamics. R represents the measurement noise covariance, which models how noisy the sensor measurement is.\newline

Together, these matrices, along with the calculated outputs from the predict and measurement stages are used to calculate the Kalman gain. The gain is a measure of the difference between the actual and predicted measurements, and is modulated accordingly using the difference between the predicted and measured states. This is then used to update the state of the system.

\end{document}