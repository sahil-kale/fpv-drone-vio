\documentclass[bare_jrnl_transmag]{subfiles}
\begin{document}
\subsection{Madgwick Filter Background}
The Madgwick filter is an orientation filter that uses quaternions to describe the orientation from a fixed body frame to a moving inertial frame \cite{ahrs_madgwick}. As a result of the algorithms using quaternions, it does not suffer from the problems presented orientation representation as Euler angles, such as gimbal lock. The equations of the filter (taken from \cite{ahrs_madgwick}) are restated below:

\subsubsection{Gyroscopic Integration Step}
The first step of the Madgwick filter (alongside a number of other inertial navigation filters) is to integrate the gyroscopic measurements (angular rate about the body frame axes, represented by $[\omega_x \ \omega_y \ \omega_z]$) to get a new estimation of the inertial frame's orientation, shown in Equation \ref{eq:madgwick_gyro_integration}.

\begin{equation}
    q_{k} = q_{k-1} + \frac{q_{k-1} \otimes [0 \ \omega_x \ \omega_y \ \omega_z] \Delta t}{2}
    \label{eq:madgwick_gyro_integration}
\end{equation}

\subsubsection{Accelerometer Correction Step}
The second step of the Madgwick filter is to correct the orientation estimate using the accelerometer measurements. The accelerometer measurements are represented in the body frame as $[a_x \ a_y \ a_z]$, with the goal of the correction step to align the accelerometer measurements, converted to the world frame, with the gravity vector in the world frame, represented as $[0 \ 0 \ g]$ (though normalized to $[0 \ 0 \ 1]$ to simplify the equations - note the gravity vector pointing up is due to how the sensor frame is oriented). From this, an optimization problem can be formed to minimize the difference between the two vectors, shown in Equations \ref{eq:madgwick_optimization_function} and \ref{eq:madgwick_optimization_function_jacobian}.

\begin{equation}
f_g(q, s_a) =
\begin{bmatrix}
    2(q_x q_z - q_w q_y) - a_x \\
    2(q_w q_x + q_y q_z) - a_y \\
    2\left(\frac{1}{2} - q_x^2 - q_y^2\right) - a_z
    \end{bmatrix}
    \label{eq:madgwick_optimization_function}
\end{equation}
    
\begin{equation}
    J_g(q) =
    \begin{bmatrix}
    -2q_y & 2q_z & -2q_w & 2q_x \\
    2q_x & 2q_w & 2q_z & 2q_y \\
    0 & -4q_x & -4q_y & 0
\end{bmatrix}
    \label{eq:madgwick_optimization_function_jacobian}
\end{equation}

The gradient of the optimization function is calculated with the following Equation (\ref{eq:madgwick_optimization_function_gradient}):
\begin{equation}
    \nabla f_g(q, s_a) = J_g(q)^T f_g(q, s_a)
    \label{eq:madgwick_optimization_function_gradient}
\end{equation}

A correctional step is then performed using the gradient of the optimization function, as shown in Equation \ref{eq:madgwick_correction_step}. Note the step size $\mu$ is a tunable parameter that can be adjusted to change the aggressiveness of the correction step.
\begin{equation}
    q_{k} = q_{k} - \mu \frac{\nabla f_g(q, s_a)}{||\nabla f_g(q, s_a)||}\Delta t
    \label{eq:madgwick_correction_step}
\end{equation}

\end{document}